

%%% This LaTeX source document can be used as the basis for your technical
%%% report. Intentionally stripped and simplified
%%% and commands should be adjusted for your particular paper - title, 
%%% author, citations, equations, etc.
% % Citations/references are in report.bib 

\documentclass[conference,backref=page]{acmsiggraph}
\usepackage{url}

\TOGonlineid{45678}
\TOGvolume{0}
\TOGnumber{0}
\TOGarticleDOI{1111111.2222222}
\TOGprojectURL{}
\TOGvideoURL{}
\TOGdataURL{}
\TOGcodeURL{}

% Include this so that citations show up in blue and the page information is included in the reference section
\hypersetup{
    colorlinks = true, 
    linkcolor = blue,
    anchorcolor = red,
    citecolor = blue, 
    filecolor = red, 
}


\title{Simulation of Objects and Their Interaction in a Dynamic 3D Environment\\
	   Final Report Document}

\author{Beej Persson \thanks{e-mail:40183743@live.napier.ac.uk} \\
Edinburgh Napier University\\
Physics-Based Animation (SET09119)}
\pdfauthor{Beej Persson}

\keywords{physics-based, animation, simulation, cloth, forces, rigid body, collisions, mesh, springs}

\begin{document}

\teaser{
   \includegraphics[height=1.5in]{images/sampleteaser}
   \caption{(a) Static scene with target cam, (b) Collisions displayed using rotating cam, (c) free cam demonstration, (d) rigid body sphere collisions.}
   \label{teaser}
 }

\maketitle

\raggedbottom

\begin{abstract}

This document looks to outline the successes and failures of the proposed simulation idea and compare it to the blueprint provided by the design document. The core idea of the intended design was to simulate a directional force being applied to a cloth-like structure. The desired final aesthetic of an interactive leaf-blower directed towards the cloth was not achieved. The simulation is now a simple demonstration of some interactive objects and their interaction with an emphasis on collisions.

\end{abstract}



\keywordlist

%% Use this only if you're preparing a technical paper to be published in the 
%% ACM 'Transactions on Graphics' journal.

% \TOGlinkslist

% \copyrightspace


\section{Introduction}


%notes

\paragraph{What problem are you solving?}
Physically simulating natural looking cloth-like structures has often been one of the harder aspects in games, especially if interaction with the cloth is required (a character walking through drapes in a doorway, or sheets on a washing line). For the most part they are simply animated to move in a way such that it looks as though it is reacting to its environment, effected by the wind or a characters motion, when in fact it isn't. This is in part due to the difficulty of accurately simulating cloth physics without significantly affecting performance. This project will attempt to deal with these problems.


\paragraph{What is the motivation?}
If cloth physics can be accurately simulated, 3D game environments can take a significant step in the direction of realism and immersion. With the Virtual Reality industry expanding as fast as it is any increase in immersion is highly desired.

\paragraph{Challenge \& Limitations}
Simulating the physics of cloths can be a very difficult process due to the "incredibly complex interactions that cloth has on the macro and microscopic levels" \cite{kristopher}. There are three main approaches to simulating cloths, geometric, physical and particle based techniques, and each has their own issues. Geometric basic techniques work well for single-frame renders but not for dynamic models \cite{Weil}. Physical methods are better at dynamically modelling cloth but still don't take into account micro-interactions happening within the cloth \cite{Feynman}. Finally, particle based modelling better accounts for the complex interactions happening within the interior of cloth structures but is the most taxing on performance \cite{Breen}.

\paragraph{Our Work}
The intended technique to be used in the proposed simulation idea was to model the cloth as a system of interconnected particles, following the work of David Breen and a tutorial based off of what he's done. This was limited by the amount of time available for the project, and simple physical techniques were implemented instead.

\section{Related Work}
The most famous initial piece of work on cloth physics was Weil's geometric approach \cite{Weil} in which he represented the cloth as a grid of multiple points. Weil's work forms the basis for almost all other cloth simulation techniques \cite{Karthikeyan}. Feynman in 1986 also modelled the cloth using a grid, but using Equation \ref{eq:physical} below:

\begin{equation} \label{eq:physical}
	E(Particle_{i,j}) = k_{s}E_{s,i,j} + k_{b}E_{b,i,j} + k_{g}E_{g,i,j}
\end{equation}

Where $s$ is Elasticity, $b$ is its bendiness and $g$ its density, Feynman observed that the final shape of the cloth is created when the energy of the cloth is at a minimum. Thus creating a way for modelling cloths that accounted for a materials stretch, stiffness and weight. David Breen, one of the foremost people who has worked in modelling cloths using particle systems, has attempted many different forms of the particle based modelling technique over the years. A general approach to the particle model is to suppose that there is a network of particles interacting with each other and by modelling the properties of this micro-structure and their interactions a realistic motion in the cloth is simulated. This is shown by Equation \ref{eq:particle} below:

\begin{equation} \label{eq:particle}
U_{Total} = U_{Repel} + U_{Stretch} + U_{Bend} + U_{Trellis} + U_{Gravity}
\end{equation}

Where $U$ is the energy of the respective elements of the system.

% SIMULATION
\section{Simulation}

\paragraph{Overview}
The core principle of the simulation is a look at objects and their interaction. The system involves a number of interactive particle-like spheres and rigid-body cubes reacting to collisions with each other whilst under the effects of gravity. Mechanically an interactive sphere that has a linear force added moving it around the scene to collide with the other objects.

\paragraph{Detailed description}
\begin{itemize}
\item {\bf Functionality}: The simulation sees the effects of a collision between particles and attempts to animate a realistic reaction to this collision. Multiple cameras were added, and the ability to toggle between them using the keyboard. A free camera (using WASD and the mouse for movement), a target camera and a rotating camera were all added to view the scene. The objects rendered in the scene can also be toggled between two different sets of objects.
\item {\bf Control/interactivity}: A sphere can be moved around the scene and collided with objects. The cubes can have angular momentum applied to them using the 'space' key.
\end{itemize}

\paragraph{Implementation}

\begin{itemize}
\item  {\bf Major Software Development Tasks} The primary difficulty was correctly implementing collisions and ensuring they looked realistic. Developing the cloth physics would have likely been even more difficult to implement.
\item {\bf Risks} As discussed in the design document initially, the modelling of the cloth-like structure did prove too difficult and as such simpler physical implementations were pursued.
\end{itemize}

% EVALUATION
\section{Testing and evaluation}
Certain tests were carried out to ensure that the simulation worked within the desired environment. For example, testing whether or not the objects reacted in a visually realistic way to the effect of being hit by the sphere. Tests were also run using volunteers to determine if the design of the interactivity is intuitive and easy to use.


\section{Conclusion and Future work}
The desired design of the proposed simulation idea was to see the effects of focused air on a cloth-like structure with certain nodes locked in place (for example the top nodes for the rail of a curtain). There are three main modelling approaches: geometric, physical and particle based systems. The intended method to be used was a particle based model where the cloth's micro-structure is modelled by having a network of particles interacting with each other and determining the desired position of each particle (and therefore shape of the cloth) by using the energy Equation \ref{eq:particle} shown above. However, due to the challenges that presented themselves, this was not achieved. As such a lot of future work could still be done.


% \section*{Acknowledgements}


\bibliographystyle{acmsiggraph}
\bibliography{report}

\end{document}

